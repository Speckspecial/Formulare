\documentclass[a4paper, 10pt]{article}
\pdfminorversion=7

\usepackage[margin=1.5cm]{geometry}
\usepackage{graphicx}
\usepackage{titling}
\usepackage{setspace}
\usepackage{makecell}
\usepackage{xcolor}
\usepackage[most]{tcolorbox}
\usepackage{ocgx2}      % erzeugt OCG-Layer in LaTeX
\usepackage{tikz}
\usepackage{eforms}
\usepackage{insdljs}    % eingebettetes JavaScript

\hypersetup{ 
    pdfauthor={}, 
    pdftitle={SV 1919 Osterburken e.V. Mitgliedsantrag (Stand 02.2026)}, 
    pdfcreator={LaTeX}, 
    pdfproducer={pdfLaTeX}
}
\newcommand{\DateField}[3]{
\begin{ocg}{ocg#1}{ocg#1}{false}
    \setlength{\fboxsep}{0.02cm}
    \setlength{\fboxrule}{1pt}
    \fcolorbox{red}{white}{
        \TextField[
            onblur={dateChecker(#2, #3);}, % range in month (negativ are in past, positiv are in futute)
            validate={dateChecker(#2, #3);}, % range in month (negativ are in past, positiv are in futute)
            name=#1,
            format={AFDate_FormatEx("dd.mm.yyyy");},
            keystroke={AFDate_KeystrokeEx("dd.mm.yyyy");},
            bordercolor={0 0 0},
            borderwidth=1pt,
            borderstyle=S,
            height=\editheight,
            width=3cm,
        ]{}
    }
\end{ocg}
}

\newcommand{\EmailField}[1]{
\begin{ocg}{ocg#1}{ocg#1}{false}
    \setlength{\fboxsep}{0.02cm}
    \setlength{\fboxrule}{1pt}
    \fcolorbox{red}{white}{
        \TextField[
            name=#1,
            keystroke={emailKeystroke();},
            onblur={emailChecker();},
            validate={emailChecker();},
            bordercolor={0 0 0},
            width=10cm,
            height=\editheight
        ]{}
    }
\end{ocg}
}

\newcommand{\PhoneField}[1]{
\begin{ocg}{ocg#1}{ocg#1}{false}
    \setlength{\fboxsep}{0.02cm}
    \setlength{\fboxrule}{1pt}
    \fcolorbox{red}{white}{
        \TextField[
            name=#1,
            keystroke={phoneKeystroke();},
            onblur={phoneChecker();},
            validate={phoneChecker();},
            bordercolor={0 0 0},
            width=10cm,
            height=\editheight
        ]{}
    }
\end{ocg}
}

\newcommand{\IBANField}[1]{
\begin{ocg}{ocg#1}{ocg#1}{false}
    \setlength{\fboxsep}{0.02cm}
    \setlength{\fboxrule}{1pt}
    \fcolorbox{red}{white}{
        \TextField[
            name=#1,
            keystroke={ibanKeystroke();},
            onblur={ibanChecker();},
            validate={ibanChecker();},
            format={var f = this.getField(event.targetName); f.textFont="Courier";},
            bordercolor={0 0 0},
            width=8cm,
            height=\editheight
        ]{}
    }
\end{ocg}
}

\newcommand{\internalpersonneutral}[4]{
\begin{tabular}{@{}p{2.5cm} @{}l}
\makecell[cl]{Vorname:} &
\makecell[cl]{\TextField[
    name=txtVorname#2,
    bordercolor={0 0 0},
    width=10cm, height=\editheight
]{}} \\[3ex]
\makecell[cl]{Nachname:} &
\makecell[cl]{\TextField[
    name=txtNachname#2,
    bordercolor={0 0 0},
    width=10cm, height=\editheight
]{}} \\[3ex]
\makecell[cl]{Geburtsdatum:} &
\makecell[cl]{\DateField{txtGeburtsdatum#2}{-#4}{-#3}}
\end{tabular}
}

\newcommand{\personneutral}[2]{
    \internalpersonneutral{#1}{#2}{216}{1320}
}

\newcommand{\internalperson}[4]{
\internalpersonneutral{#1}{#2}{#3}{#4}
\\[2ex]
\begin{tabular}{@{}p{2.5cm} @{}l @{\extracolsep{0.5em}}l l @{\extracolsep{0.5em}}l l @{\extracolsep{0.5em}}l l @{\extracolsep{0.5em}}l}
\makecell[cl]{Geschlecht:} &
\makecell[cl]{\ChoiceMenu[radio,name=radioGeschlecht#2, radiosymbol=\ding{108}, bordercolor={0 0 0}]{}{=m}} &
\makecell[cl]{männlich} &
\makecell[cl]{\ChoiceMenu[radio,name=radioGeschlecht#2, radiosymbol=\ding{108}, bordercolor={0 0 0}]{}{=w}} &
\makecell[cl]{weiblich} &
\makecell[cl]{\ChoiceMenu[radio,name=radioGeschlecht#2, radiosymbol=\ding{108}, bordercolor={0 0 0}]{}{=d}} &
\makecell[cl]{divers} &
\makecell[cl]{\ChoiceMenu[radio,name=radioGeschlecht#2, radiosymbol=\ding{108}, bordercolor={0 0 0}]{}{=u}} &
\makecell[cl]{keine Angabe}
\end{tabular}
}

\newcommand{\person}[2]{
    \internalperson{#1}{#2}{216}{1320}
}

\newcommand{\address}[1]{
\noindent
\\[1ex]
\begin{tabular}{@{}p{1cm} l l l@{}}
\makecell[cl]{Straße:} &
\makecell[cl]{\TextField[
    name=txtStrasse#1,
    width=10cm,
    height=\editheight,
    bordercolor={0 0 0}
]{}} &
\makecell[cl]{Hausnr.:} &
\makecell[cl]{\TextField[
    name=txtHausNr#1,
    width=2cm,
    height=\editheight,
    bordercolor={0 0 0}
]{}}
\end{tabular}

\noindent
\\[1ex]
\begin{tabular}{@{}p{1cm} l l l l l@{}}
\makecell[cl]{PLZ:} &
\makecell[cl]{\TextField[
    name=txtPLZ#1,
    width=2.4cm,
    height=\editheight,
    bordercolor={0 0 0}
]{}} &
\makecell[cl]{Ort:} &
\makecell[cl]{\TextField[
    name=txtOrt#1,
    width=6.2cm,
    height=\editheight,
    bordercolor={0 0 0}
]{}} &
\makecell[cl]{Land:} &
\makecell[cl]{\TextField[
    name=txtLand#1,
    width=3.5cm,
    height=\editheight,
    bordercolor={0 0 0}
]{}}
\end{tabular}
}

\newcommand{\department}[1]{
\begin{tabular}{@{}p{2.5cm} @{}l @{\extracolsep{0.5em}}l l @{\extracolsep{0.5em}}l}
\makecell[cl]{Abteilung:} &
\makecell[cl]{\ChoiceMenu[radio,name=radioAbteilung#1, radiosymbol=\ding{108}, bordercolor={0 0 0}]{}{=f}} &
\makecell[cl]{Fußball} &
\makecell[cl]{\ChoiceMenu[radio,name=radioAbteilung#1, radiosymbol=\ding{108}, bordercolor={0 0 0}]{}{=t}} &
\makecell[cl]{Turnen}
\end{tabular}
}

\newcommand{\internalmember}[6]{
\begin{tcolorbox}[
    colback=white,    % Hintergrundfarbe
    colframe=#4,  % Rahmenfarbe
    title={#3},  % Der Titel
    fonttitle=\bfseries,     % Titel fett
    arc=1.5mm,           % Ecken (optional: arc=2mm für gerundet)
    boxrule=0.5mm            % Dicke des Rahmens
]
\internalperson{#1}{#2}{#5}{#6}
\noindent
\\[2ex]
\department{#2}
\end{tcolorbox}
}

\newcommand{\member}[4]{
    \internalmember{#1}{#2}{#3}{#4}{216}{1320} % age <= 1320 month (= 120 years)
}

\newcommand{\memberchild}[4]{
    \internalmember{#1}{#2}{#3}{#4}{0}{216} % age <= month (= 18 years)
}

\newcommand{\headermembers}{
\begin{flushleft}
    \vspace*{-1.2cm}
    \begin{minipage}{0.8\textwidth}
        {\Huge\bfseries\textcolor{svoblau}{SV~1919~Osterburken~e.V.}}
        \\[1ex]{\Large AUFNAHMEANTRAG (Stand 02/2026) \hfill Seite \thepage}
        \\[4ex]
        Mitgliederdaten (ggf. leere Seite)
    \end{minipage}
    \hfill
    \begin{minipage}{0.15\textwidth}
        \raggedleft
        \includegraphics[width=\linewidth]{SVO Logo 2019.jpg} % <-- Dein Logo hier
    \end{minipage}
\end{flushleft}
}

\newcommand{\signature}[1]{
\begin{tabular}{@{}p{2cm} @{}l}
\makecell[cl]{Ort:} &
\makecell[cl]{\TextField[
    name=txtOrt#1,
    bordercolor={0 0 0},
    width=5cm, height=\editheight
]{}
}\\[3ex]
\makecell[cl]{Datum:} &
\makecell[cl]{\DateField{txtDatum#1}{-12}{12}}
\end{tabular}\hfill
\begin{tabular}{l}
\makecell[cc]{\sigField[\BC{0,0,0}]{sigUnterschrift#1}{9cm}{1.8cm}\\[0.5ex]
\rule{9cm}{0.4pt}\\
\footnotesize Unterschrift}
\end{tabular}
}

%\makecell[cc]{\rule{6cm}{0.4pt}\\[0.5ex]
%    \footnotesize Unterschrift
%}


\definecolor{svoblau}{RGB}{24,47,149}

\input{eventhandler.js}

% ---------------------------------------------------------
% Dokument
% ---------------------------------------------------------
\pagenumbering{arabic}
\setlength{\parindent}{0pt}


\begin{document}
\begin{Form}[NeedAppearances=true]

\begin{flushleft}
    \vspace*{-1.2cm}
    \begin{minipage}{0.8\textwidth}
        {\Huge\bfseries\textcolor{svoblau}{SV~1919~Osterburken~e.V.}}
        \\[1ex]{\Large AUFNAHMEANTRAG (Stand 02.2026) \hfill Seite \thepage}
        \\[4ex]
        Hiermit beantrage ich die Mitgliedschaft beim SV~1919~Osterburken~e.V..
    \end{minipage}
    \hfill
    \begin{minipage}{0.15\textwidth}
        \raggedleft
        \includegraphics[width=\linewidth]{SVO Logo 2019.jpg} % <-- Dein Logo hier
    \end{minipage}
\end{flushleft}

\begin{tcolorbox}[
    colback=white,    % Hintergrundfarbe
    colframe=svoblau,  % Rahmenfarbe
    title={Antragsteller/-in},  % Der Titel
    fonttitle=\bfseries,     % Titel fett
    arc=1.5mm,           % Ecken (optional: arc=2mm für gerundet)
    boxrule=0.5mm            % Dicke des Rahmens
]
\begin{tabular}{@{}l l@{}}
\makecell[cl]{Eintrittsdatum:} &
\makecell[cl]{\DateField{txtEintrittsdatum}{-12}{12}}
\end{tabular}
\\[1ex]
\begin{tabular}{@{}l l@{}}
\makecell[cl]{\ChoiceMenu[
    name=radioMitgliedschaft,
    radio, radiosymbol=\ding{108},
    keystroke={selectMemberType("memberType");},
    bordercolor={0 0 0}
]{}{=E}} &
\makecell[cl]{Mitgliedschaft für mich (ab 18 Jahre alt) (69,- Euro/Jahr)}\\[2ex]
\makecell[cl]{\ChoiceMenu[
    name=radioMitgliedschaft,
    radio, radiosymbol=\ding{108},
    keystroke={selectMemberType("memberType");},
    bordercolor={0 0 0}
]{}{=K}} &
\makecell[cl]{Mitgliedschaft für mein Kind (bis 17 Jahre alt) (59,- Euro/Jahr)\\
\bfseries Antragsteller/-in ist erziehungsberechtigt.}\\[2ex]
\makecell[cl]{\ChoiceMenu[
    name=radioMitgliedschaft,
    radio, radiosymbol=\ding{108},
    keystroke={selectMemberType("memberType");},
    bordercolor={0 0 0}
]{}{=F}} &
\makecell[cl]{Familienmitgliedschaft für mich, Partner/Partnerin und Kinder (110,- Euro/Jahr)\\
\bfseries Antragsteller/-in ist Familienmitglied und erziehungsberechtigt.}\\
\end{tabular}

\noindent
\\[2ex]
\person{Antragsteller/-in}{Antragsteller}

\address{Antragsteller}

\noindent
\\[2ex]
\begin{tabular}{@{}l l@{}}
\makecell[cl]{E-Mail:} &
\makecell[cl]{\EmailField{txtEMail}}\\[3ex]
\makecell[cl]{Telefon:} &
\makecell[cl]{\PhoneField{txtTelefon}}
\end{tabular}

\end{tcolorbox}

\begin{ocg}{AntragstellerAbteilung}{AntragstellerAbteilung}{true}
\begin{tcolorbox}[
    colback=white,    % Hintergrundfarbe
    colframe=svoblau,  % Rahmenfarbe
    title={Abteilung für Antragsteller/-in als Mitglied},  % Der Titel
    fonttitle=\bfseries,     % Titel fett
    arc=1.5mm,           % Ecken (optional: arc=2mm für gerundet)
    boxrule=0.5mm            % Dicke des Rahmens
]
\noindent
\\[0.5ex]
\department{Antragsteller}
\end{tcolorbox}
\end{ocg}

\begin{tcolorbox}[
    colback=white,    % Hintergrundfarbe
    colframe=svoblau,  % Rahmenfarbe
    fonttitle=\bfseries,     % Titel fett
    arc=1.5mm,           % Ecken (optional: arc=2mm für gerundet)
    boxrule=0.5mm            % Dicke des Rahmens
]
\section*{Einverständniserklärung}

Ich erkenne die Vereinssatzung in allen Rechten und Pflichten an.
Ich bin damit einverstanden, Informationen, Einladungen etc. des Vereins per E-Mail zu erhalten.

\noindent
\\[1ex]
\signature{Einverstaendnis}
\end{tcolorbox}

\pagebreak
\headermembers

\begin{ocg}{MeinKind}{MeinKind}{true}
\memberchild{Mein Kind}{Kind}{Einzelmitgliedschaft für mein Kind}
\end{ocg}

\begin{ocg}{Familie}{Familie}{true}
\vspace{0.5cm}
\begin{tcolorbox}[
    colback=white,    % Hintergrundfarbe
    colframe=svoblau,  % Rahmenfarbe
    title={Mitgliedschaft für die gesante Familie},  % Der Titel
    fonttitle=\bfseries,     % Titel fett
    arc=1.5mm,           % Ecken (optional: arc=2mm für gerundet)
    boxrule=0.5mm            % Dicke des Rahmens
]
\begin{tabular}{@{}l l @{\hspace{0.5em}}l l @{\hspace{0.5em}}l}
\makecell[cl]{Familienmitgliedschaft mit} &
\makecell[cl]{\CheckBox[
    name=checkPartner,
    bordercolor={0 0 0},
    keystroke={selectMemberType("checkPartner");}
]{}} &
\makecell[cl]{Partner/-in und} &
\makecell[cl]{\ChoiceMenu[
    name=choiceKinder,
    radio,
    keystroke={selectMemberType("choiceKinder");},
    bordercolor={0 0 0},
    radiosymbol=\ding{108}
]{}{{ 0 }=0, { 1 }=1, { 2 }=2, { 3 }=3, { 4 }=4}Kind/Kinder}
\end{tabular}
\end{tcolorbox}
\end{ocg}

\begin{ocg}{Partner}{Partner}{true}
\member{Partner/-in}{Partner}{Familienmitgliedschaft für den/die Partner/-in}
\end{ocg}

\begin{ocg}{Kind0}{Kind0}{true}
\memberchild{1. Kind}{Kind0}{Familienmitgliedschaft für das 1. Kind}
\end{ocg}

\pagebreak
\headermembers

\begin{ocg}{Kind1}{Kind1}{true}
\memberchild{2. Kind}{Kind1}{Familienmitgliedschaft für das 2. Kind}
\end{ocg}

\begin{ocg}{Kind2}{Kind2}{true}
\memberchild{3. Kind}{Kind2}{Familienmitgliedschaft für das 3. Kind}
\end{ocg}

\begin{ocg}{Kind3}{Kind3}{true}
\memberchild{4. Kind}{Kind3}{Familienmitgliedschaft für das 4. Kind}
\end{ocg}

\pagebreak

\begin{flushleft}
    \vspace*{-1.2cm}
    \begin{minipage}{0.8\textwidth}
        {\Huge\bfseries\textcolor{svoblau}{SV~1919~Osterburken~e.V.}}
        \\[1ex]{\Large AUFNAHMEANTRAG (Stand 02/2026) \hfill Seite \thepage}
        \\[4ex]
        {\Large\bfseries SEPA Lastschriftmandat}
    \end{minipage}
    \hfill
    \begin{minipage}{0.15\textwidth}
        \raggedleft
        \includegraphics[width=\linewidth]{SVO Logo 2019.jpg} % <-- Dein Logo hier
    \end{minipage}
\end{flushleft}


\begin{tcolorbox}[
    colback=white,    % Hintergrundfarbe
    colframe=svoblau,  % Rahmenfarbe
    fonttitle=\bfseries,     % Titel fett
    arc=1.5mm,           % Ecken (optional: arc=2mm für gerundet)
    boxrule=0.5mm            % Dicke des Rahmens
]
\subsection*{Zahlungsempfänger}
\begin{tabular}{@{}l l l l}
\makecell[cl]{Name:} &
\makecell[cl]{\TextField[
    name=txtSVOName,
    bordercolor={0 0 0},
    readonly,
    value={Sportverein 1919 Osterburken e.V.},
    width=7cm, height=0.5cm,
  ]{}
} \\[3ex]
\makecell[cl]{Straße und Hausnr.:} &
\makecell[cl]{\TextField[
    name=txtSVOStrasseHausnr,
    bordercolor={0 0 0},
    readonly,
    value={Postfach 1244},
    width=7cm, height=0.5cm
  ]{}
} & & \\[3ex]
\makecell[cl]{PLZ und Ort:} &
\makecell[cl]{\TextField[
    name=txtSVOPLZOrt,
    bordercolor={0 0 0},
    readonly,
    value={74702 Osterburken},
    width=7cm, height=0.5cm
  ]{}
} &
\makecell[cl]{Land.} &
\makecell[cl]{\TextField[
    name=txtSVOLand,
    bordercolor={0 0 0},
    readonly,
    value={Deutschland},
    width=4cm, height=0.5cm
]{}}
\end{tabular}
\noindent
\\[2ex]
\begin{tabular}{@{}l l}
\makecell[cl]{Gläubiger-Identifikationsnummer:} &
\makecell[cl]{\TextField[
    name=txtSVOIdentnr,
    bordercolor={0 0 0},
    readonly,
    format={var f = this.getField(event.targetName); f.textFont="Courier";},
    value={ D E 5 9 Z Z Z 0 0 0 0 0 6 1 4 9 5 4},
    width=8cm, height=0.5cm
]{}} \\[2ex]
\makecell[cl]{Mandatsreferenz:\\ (vom Zahlungsempfänger auszufüllen)} &
\makecell[cl]{\TextField[
    name=txtSVOMandatref,
    bordercolor={0 0 0},
    readonly,
    width=8cm, height=0.5cm
]{}}
\end{tabular}

\noindent
\\[0.5ex]
\rule{\textwidth}{0.4pt}
\\[1ex]
Ich ermächtige / Wir ermächtigen (A) den Zahlungsempfänger (Name siehe oben), Zahlungen von meinem / unserem Konto
mittels Lastschrift einzuziehen. Zugleich (B) weise ich mein / weisen wir unser Kreditinstitut an, die vom Zahlungsempfänger
(Name siehe oben) auf mein / unser Konto gezogenen Lastschriften einzulösen.
\par\medskip
Hinweis: Ich kann / Wir können innerhalb von acht Wochen, beginnend mit dem Belastungsdatum, die Erstattung des belasteten
Betrages verlangen. Es gelten dabei die mit meinem / unserem Kreditinstitut vereinbarten Bedingungen.

\noindent
\\[0.5ex]
\begin{tabular}{@{}l l @{\hspace{0.5em}}l}
\makecell[cl]{Zahlungsart:} &
\makecell[cl]{\CheckBox[
    name=checkSVOZahlungsart,
    bordercolor={0 0 0},
    readonly,
    checked
]{}} &
\makecell[cl]{Wiederkehrende Zahlung}
\end{tabular}
\\[1ex]
\rule{\textwidth}{0.4pt}
\\[2ex]
\begin{tabular}{@{}l l @{\hspace{0.5em}}l l @{\hspace{0.5em}}l}
\makecell[cl]{Zahlungspflichtiger:} &
\makecell[cl]{\ChoiceMenu[
    radio,
    radiosymbol=\ding{108},
    name=checkSVOZahler,
    keystroke={selectPayor();},
    bordercolor={0 0 0}
]{}{=A}} &
\makecell[cl]{Antragsteller/-in} &
\makecell[cl]{\ChoiceMenu[
    radio,
    radiosymbol=\ding{108},
    name=checkSVOZahler,
    keystroke={selectPayor();},
    bordercolor={0 0 0}
]{}{=N}} &
\makecell[cl]{Andere Person}
\end{tabular}

\noindent
\\[1ex]
\personneutral{Zahlungspflichtige/r und Bankdaten}{Zahler}
\noindent
\\[2ex]
\address{Zahler}

\noindent
\\[0.5ex]
\rule{\textwidth}{0.4pt}
\\[2ex]
\begin{tabular}{@{}p{3cm} l}
\makecell[cl]{IBAN:\\ (max. 34 Stellen)} &
\makecell[cl]{\IBANField{txtIBANZahler}}
\end{tabular}

\noindent
\\[2ex]
\begin{tabular}{@{}p{3cm} l p{7cm} }
\makecell[cl]{BIC:\\ (8 oder 11 Stellen)} &
\makecell[cl]{\TextField[
    name=txtBICZahler,
    bordercolor={0 0 0},
    maxlen=11,
    format={var f = this.getField(event.targetName); f.textFont="Courier";},
    width=4cm, height=0.7cm
]{}} &
\makecell[cl]{\small Hinweis: Die Angabe des BIC kann entfallen,\\
\small wenn der Zahlungsdienstleister des/der Zahlungspflichtigen\\
\small in einem EU-/EWR-Mitgliedsstaat ansässig ist.}
\end{tabular}
\rule{\textwidth}{0.4pt}
\\[2ex]
\signature{Unterschreiber}

\end{tcolorbox}


\end{Form}
\end{document}
