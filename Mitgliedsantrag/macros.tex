
\newcommand{\person}[2]{
\subsection*{#1}
\begin{tabular}{@{}p{2.5cm} @{}l}
\makecell[cl]{Vorname:} & 
\makecell[cl]{\TextField[
    name=txtVorname#2,
    bordercolor={0 0 0}, 
    width=10cm, height=0.7cm
  ]{}
} \\[3ex]
\makecell[cl]{Nachname:} & 
\makecell[cl]{\TextField[
    name=txtNachname#2, 
    bordercolor={0 0 0}, 
    width=10cm, height=0.7cm
  ]{}
} \\[3ex]
\makecell[cl]{Geburtsdatum:} &
\makecell[cl]{\TextField[
    name=txtGeburtsdatum#2, 
    format={AFDate_FormatEx("dd.mm.yyyy");}, 
    keystroke={AFDate_KeystrokeEx("dd.mm.yyyy");}, 
    bordercolor={0 0 0},
    height=0.7cm
  ]{}
} \\
\end{tabular}
\\[1ex]
\begin{tabular}{@{}p{2.5cm} @{}l @{\extracolsep{0.5em}}l l @{\extracolsep{0.5em}}l l @{\extracolsep{0.5em}}l l @{\extracolsep{0.5em}}l}
\makecell[cl]{Geschlecht:} & 
\makecell[cl]{\ChoiceMenu[radio,name=radioGeschlecht#2, radiosymbol=\ding{55}, bordercolor={0 0 0}]{}{=m}} &
\makecell[cl]{männlich} &
\makecell[cl]{\ChoiceMenu[radio,name=radioGeschlecht#2, radiosymbol=\ding{55}, bordercolor={0 0 0}]{}{=w}} &
\makecell[cl]{weiblich} &
\makecell[cl]{\ChoiceMenu[radio,name=radioGeschlecht#2, radiosymbol=\ding{55}, bordercolor={0 0 0}]{}{=d}} &
\makecell[cl]{divers} &
\makecell[cl]{\ChoiceMenu[radio,name=radioGeschlecht#2, radiosymbol=\ding{55}, bordercolor={0 0 0}]{}{=u}} &
\makecell[cl]{keine Angabe}
\end{tabular}
}

\newcommand{\department}[1]{
\\[1ex]
\begin{tabular}{@{}p{2.5cm} @{}l @{\extracolsep{0.5em}}l l @{\extracolsep{0.5em}}l}
\makecell[cl]{Abteilung:} & 
\makecell[cl]{\ChoiceMenu[radio,name=radioAbteilung#1, radiosymbol=\ding{55}, bordercolor={0 0 0}]{}{=f}} &
\makecell[cl]{Fußball} &
\makecell[cl]{\ChoiceMenu[radio,name=radioAbteilung#1, radiosymbol=\ding{55}, bordercolor={0 0 0}]{}{=t}} &
\makecell[cl]{Turnen} 
\end{tabular}
}

\newcommand{\member}[3]{
\begin{tcolorbox}[
    colback=white,    % Hintergrundfarbe
    colframe=svoblau,  % Rahmenfarbe
    title=#3,  % Der Titel
    fonttitle=\bfseries,     % Titel fett
    arc=1.5mm,           % Ecken (optional: arc=2mm für gerundet)
    boxrule=0.5mm            % Dicke des Rahmens
]
\person{#1}{#2}

\department{#2}
\end{tcolorbox}
}

\newcommand{\headerfamily}{
\begin{flushleft}
    \vspace*{-1.2cm}
    \begin{minipage}{0.8\textwidth}
        {\Huge\bfseries\textcolor{svoblau}{SV~1919~Osterburken~e.V.}}
        \\[1ex]{\Large AUFNAHMEANTRAG (Stand 02/2026) \hfill Seite \thepage}
        \\[4ex]
        Daten zur Familienmitgliedschaft
    \end{minipage}
    \hfill
    \begin{minipage}{0.15\textwidth}
        \raggedleft
        \includegraphics[width=\linewidth]{SVO Logo 2019.jpg} % <-- Dein Logo hier
    \end{minipage}
\end{flushleft}
}

\newcommand{\signature}[1]{
\\[3ex]
\begin{tabular}{@{}l l l l l}
\makecell[cl]{Ort:} & 
\makecell[cl]{\TextField[
    name=txtOrt#1,
    bordercolor={0 0 0}, 
    width=4.5cm, height=0.7cm
  ]{}
} &
\makecell[cl]{Datum:} & 
\makecell[cl]{\TextField[
    name=txtDatum#1, 
    format={AFDate_FormatEx("dd.mm.yyyy");}, 
    keystroke={AFDate_KeystrokeEx("dd.mm.yyyy");}, 
    bordercolor={0 0 0},
    height=0.7cm
  ]{}
} & \\
& & & & \makecell[cc]{\rule{6.5cm}{0.4pt}\\[0.5ex]
    \footnotesize Unterschrift
}
\end{tabular}
}