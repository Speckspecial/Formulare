\newcommand{\DateField}[3]{
\begin{ocg}{ocg#1}{ocg#1}{false}
    \setlength{\fboxsep}{0.02cm}
    \setlength{\fboxrule}{1pt}
    \fcolorbox{red}{white}{
        \TextField[
            onblur={dateChecker(#2, #3);}, % range in month (negativ are in past, positiv are in futute)
            validate={dateChecker(#2, #3);}, % range in month (negativ are in past, positiv are in futute)
            name=#1,
            format={AFDate_FormatEx("dd.mm.yyyy");},
            keystroke={AFDate_KeystrokeEx("dd.mm.yyyy");},
            bordercolor={0 0 0},
            borderwidth=1pt,
            borderstyle=S,
            height=\editheight,
            width=3cm,
        ]{}
    }
\end{ocg}
}

\newcommand{\EmailField}[1]{
\begin{ocg}{ocg#1}{ocg#1}{false}
    \setlength{\fboxsep}{0.02cm}
    \setlength{\fboxrule}{1pt}
    \fcolorbox{red}{white}{
        \TextField[
            name=#1,
            keystroke={emailKeystroke();},
            onblur={emailChecker();},
            validate={emailChecker();},
            bordercolor={0 0 0},
            width=10cm,
            height=\editheight
        ]{}
    }
\end{ocg}
}

\newcommand{\PhoneField}[1]{
\begin{ocg}{ocg#1}{ocg#1}{false}
    \setlength{\fboxsep}{0.02cm}
    \setlength{\fboxrule}{1pt}
    \fcolorbox{red}{white}{
        \TextField[
            name=#1,
            keystroke={phoneKeystroke();},
            onblur={phoneChecker();},
            validate={phoneChecker();},
            bordercolor={0 0 0},
            width=10cm,
            height=\editheight
        ]{}
    }
\end{ocg}
}

\newcommand{\IBANField}[1]{
\begin{ocg}{ocg#1}{ocg#1}{false}
    \setlength{\fboxsep}{0.02cm}
    \setlength{\fboxrule}{1pt}
    \fcolorbox{red}{white}{
        \TextField[
            name=#1,
            keystroke={ibanKeystroke();},
            onblur={ibanChecker();},
            validate={ibanChecker();},
            format={var f = this.getField(event.targetName); f.textFont="Courier";},
            bordercolor={0 0 0},
            width=8cm,
            height=\editheight
        ]{}
    }
\end{ocg}
}

\newcommand{\internalpersonneutral}[4]{
\begin{tabular}{@{}p{2.5cm} @{}l}
\makecell[cl]{Vorname:} &
\makecell[cl]{\TextField[
    name=txtVorname#2,
    bordercolor={0 0 0},
    width=10cm, height=\editheight
]{}} \\[3ex]
\makecell[cl]{Nachname:} &
\makecell[cl]{\TextField[
    name=txtNachname#2,
    bordercolor={0 0 0},
    width=10cm, height=\editheight
]{}} \\[3ex]
\makecell[cl]{Geburtsdatum:} &
\makecell[cl]{\DateField{txtGeburtsdatum#2}{-#4}{-#3}}
\end{tabular}
}

\newcommand{\personneutral}[2]{
    \internalpersonneutral{#1}{#2}{216}{1320}
}

\newcommand{\internalperson}[4]{
\internalpersonneutral{#1}{#2}{#3}{#4}
\\[2ex]
\begin{tabular}{@{}p{2.5cm} @{}l @{\extracolsep{0.5em}}l l @{\extracolsep{0.5em}}l l @{\extracolsep{0.5em}}l l @{\extracolsep{0.5em}}l}
\makecell[cl]{Geschlecht:} &
\makecell[cl]{\ChoiceMenu[radio,name=radioGeschlecht#2, radiosymbol=\ding{108}, bordercolor={0 0 0}]{}{=m}} &
\makecell[cl]{männlich} &
\makecell[cl]{\ChoiceMenu[radio,name=radioGeschlecht#2, radiosymbol=\ding{108}, bordercolor={0 0 0}]{}{=w}} &
\makecell[cl]{weiblich} &
\makecell[cl]{\ChoiceMenu[radio,name=radioGeschlecht#2, radiosymbol=\ding{108}, bordercolor={0 0 0}]{}{=d}} &
\makecell[cl]{divers} &
\makecell[cl]{\ChoiceMenu[radio,name=radioGeschlecht#2, radiosymbol=\ding{108}, bordercolor={0 0 0}]{}{=u}} &
\makecell[cl]{keine Angabe}
\end{tabular}
}

\newcommand{\person}[2]{
    \internalperson{#1}{#2}{216}{1320}
}

\newcommand{\address}[1]{
\noindent
\\[1ex]
\begin{tabular}{@{}p{1cm} l l l@{}}
\makecell[cl]{Straße:} &
\makecell[cl]{\TextField[
    name=txtStrasse#1,
    width=10cm,
    height=\editheight,
    bordercolor={0 0 0}
]{}} &
\makecell[cl]{Hausnr.:} &
\makecell[cl]{\TextField[
    name=txtHausNr#1,
    width=2cm,
    height=\editheight,
    bordercolor={0 0 0}
]{}}
\end{tabular}

\noindent
\\[1ex]
\begin{tabular}{@{}p{1cm} l l l l l@{}}
\makecell[cl]{PLZ:} &
\makecell[cl]{\TextField[
    name=txtPLZ#1,
    width=2.4cm,
    height=\editheight,
    bordercolor={0 0 0}
]{}} &
\makecell[cl]{Ort:} &
\makecell[cl]{\TextField[
    name=txtOrt#1,
    width=6.2cm,
    height=\editheight,
    bordercolor={0 0 0}
]{}} &
\makecell[cl]{Land:} &
\makecell[cl]{\TextField[
    name=txtLand#1,
    width=3.5cm,
    height=\editheight,
    bordercolor={0 0 0}
]{}}
\end{tabular}
}

\newcommand{\department}[1]{
\begin{tabular}{@{}p{2.5cm} @{}l @{\extracolsep{0.5em}}l l @{\extracolsep{0.5em}}l}
\makecell[cl]{Abteilung:} &
\makecell[cl]{\ChoiceMenu[radio,name=radioAbteilung#1, radiosymbol=\ding{108}, bordercolor={0 0 0}]{}{=f}} &
\makecell[cl]{Fußball} &
\makecell[cl]{\ChoiceMenu[radio,name=radioAbteilung#1, radiosymbol=\ding{108}, bordercolor={0 0 0}]{}{=t}} &
\makecell[cl]{Turnen}
\end{tabular}
}

\newcommand{\internalmember}[6]{
\begin{tcolorbox}[
    colback=white,    % Hintergrundfarbe
    colframe=#4,  % Rahmenfarbe
    title={#3},  % Der Titel
    fonttitle=\bfseries,     % Titel fett
    arc=1.5mm,           % Ecken (optional: arc=2mm für gerundet)
    boxrule=0.5mm            % Dicke des Rahmens
]
\internalperson{#1}{#2}{#5}{#6}
\noindent
\\[2ex]
\department{#2}
\end{tcolorbox}
}

\newcommand{\member}[4]{
    \internalmember{#1}{#2}{#3}{#4}{216}{1320} % age <= 1320 month (= 120 years)
}

\newcommand{\memberchild}[4]{
    \internalmember{#1}{#2}{#3}{#4}{0}{216} % age <= month (= 18 years)
}

\newcommand{\headermembers}{
\begin{flushleft}
    \vspace*{-1.2cm}
    \begin{minipage}{0.8\textwidth}
        {\Huge\bfseries\textcolor{svoblau}{SV~1919~Osterburken~e.V.}}
        \\[1ex]{\Large AUFNAHMEANTRAG (Stand 02/2026) \hfill Seite \thepage}
        \\[4ex]
        Mitgliederdaten (ggf. leere Seite)
    \end{minipage}
    \hfill
    \begin{minipage}{0.15\textwidth}
        \raggedleft
        \includegraphics[width=\linewidth]{SVO Logo 2019.jpg} % <-- Dein Logo hier
    \end{minipage}
\end{flushleft}
}

\newcommand{\signature}[1]{
\begin{tabular}{@{}p{2cm} @{}l}
\makecell[cl]{Ort:} &
\makecell[cl]{\TextField[
    name=txtOrt#1,
    bordercolor={0 0 0},
    width=5cm, height=\editheight
]{}
}\\[3ex]
\makecell[cl]{Datum:} &
\makecell[cl]{\DateField{txtDatum#1}{-12}{12}}
\end{tabular}\hfill
\begin{tabular}{l}
\makecell[cc]{\sigField[\BC{0,0,0}]{sigUnterschrift#1}{9cm}{1.8cm}\\[0.5ex]
\rule{9cm}{0.4pt}\\
\footnotesize Unterschrift}
\end{tabular}
}

%\makecell[cc]{\rule{6cm}{0.4pt}\\[0.5ex]
%    \footnotesize Unterschrift
%}
