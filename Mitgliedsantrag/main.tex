\documentclass[a4paper, 10pt]{article}
\pdfminorversion=7

\usepackage[margin=1.5cm]{geometry}
\usepackage{graphicx}
\usepackage{titling}
\usepackage{setspace}
\usepackage{makecell}
\usepackage[unicode]{hyperref}
\usepackage{xcolor}
\usepackage{ocgx2}      % erzeugt OCG-Layer in LaTeX
\usepackage{eforms}     
\usepackage{insdljs}    % eingebettetes JavaScript

\definecolor{svoblau}{RGB}{24,47,149}

\begin{insDLJS}[toggleOCG]{toggleOCG}{}
function toggleOCGByName(layerName) {
    var ocgs = this.getOCGs();
    if (!ocgs) return;

    for (var i = 0; i < ocgs.length; i++) {
        ocgs[i].state = !ocgs[i].state;
    }
}
\end{insDLJS}


% ---------------------------------------------------------
% Logo oben rechts
% ---------------------------------------------------------
\pretitle{%
  \begin{flushleft}
    \vspace*{-2cm}
    \begin{minipage}{0.7\textwidth}
}
\posttitle{%
    \end{minipage}
    \hfill
    \begin{minipage}{0.25\textwidth}
      \raggedleft
      \includegraphics[width=\linewidth]{SVO Logo 2019.jpg} % <-- Dein Logo hier
    \end{minipage}
  \end{flushleft}
  \vspace{1ex}
}

% ---------------------------------------------------------
% Dokument
% ---------------------------------------------------------
\title{\Huge\bfseries\textcolor{svoblau}{SV~1919~Osterburken~e.V.}
\\[1ex]\Large AUFNAHMEANTRAG \hfill (Stand 02/2026)
}
\author{} % kein Autor
\date{}   % kein Datum

\begin{document}

\maketitle

\section*{Antragsteller/-in}

Hiermit beantrage ich die Mitgliedschaft beim SV~1919~Osterburken~e.V..
\\[1ex]
\noindent
\begin{tabular}{@{}l l@{}}
\makecell[cl]{Eintrittsdatum:} & 
\makecell[cl]{\TextField[name=txtEintrittsdatum, 
  format={AFDate_FormatEx("dd.mm.yyyy");}, 
  keystroke={AFDate_KeystrokeEx("dd.mm.yyyy");}, 
  bordercolor={0 0 0},
  height=0.7cm
]{}}
\end{tabular}
\\[1ex]
\noindent
\begin{tabular}{@{}l l@{}}
\ChoiceMenu[name=radioMitgliedschaft, radio, radiosymbol=\ding{55}, onclick={this.getField("Einzelmitglied").display = display.visible;}, bordercolor={0 0 0}]{}{=EinzelErwachsener} & 
Mitgliedschaft für mich \\
\ChoiceMenu[name=radioMitgliedschaft, radio, radiosymbol=\ding{55}, onclick={this.getField("Einzelmitglied").display = display.hidden;}, bordercolor={0 0 0}]{}{=EinzelKind} & 
Mitgliedschaft für mein Kind (Antragsteller/-in ist erziehungsberechtigt.)\\
\ChoiceMenu[name=radioMitgliedschaft, radio, radiosymbol=\ding{55}, onclick={this.getField("Einzelmitglied").display = display.hidden;}, bordercolor={0 0 0}]{}{=Familie} & 
Familienmitgliedschaft für mich, Partner/Partnerin und Kinder \\
\end{tabular}

\subsection*{Antragsteller/Antragstellerin}

\noindent
\begin{tabular}{@{}p{2.5cm} @{}l}
\makecell[cl]{Vorname:} & 
\makecell[cl]{\TextField[
    name=txtAntragstellerVorname,
    bordercolor={0 0 0}, 
    width=10cm, height=0.7cm
  ]{}
} \\[3ex]
\makecell[cl]{Nachname:} & 
\makecell[cl]{\TextField[
    name=txtAntragstellerNachname, 
    bordercolor={0 0 0}, 
    width=10cm, height=0.7cm
  ]{}
} \\[3ex]
\makecell[cl]{Geburtsdatum:} &
\makecell[cl]{\TextField[
    name=txtAntragstellerGeburtsdatum, 
    format={AFDate_FormatEx("dd.mm.yyyy");}, 
    keystroke={AFDate_KeystrokeEx("dd.mm.yyyy");}, 
    bordercolor={0 0 0},
    height=0.7cm
  ]{}
} \\
\end{tabular}
\\[1ex]
\noindent
\begin{tabular}{@{}p{2.5cm} @{}l @{\extracolsep{0.5em}}l l @{\extracolsep{0.5em}}l l @{\extracolsep{0.5em}}l l @{\extracolsep{0.5em}}l}
\makecell[cl]{Geschlecht:} & 
\makecell[cl]{\ChoiceMenu[radio,name=radioGeschlecht, radiosymbol=\ding{55}, bordercolor={0 0 0}]{}{=m}} &
\makecell[cl]{männlich} &
\makecell[cl]{\ChoiceMenu[radio,name=radioGeschlecht, radiosymbol=\ding{55}, bordercolor={0 0 0}]{}{=w}} &
\makecell[cl]{weiblich} &
\makecell[cl]{\ChoiceMenu[radio,name=radioGeschlecht, radiosymbol=\ding{55}, bordercolor={0 0 0}]{}{=d}} &
\makecell[cl]{divers} &
\makecell[cl]{\ChoiceMenu[radio,name=radioGeschlecht, radiosymbol=\ding{55}, bordercolor={0 0 0}]{}{=u}} &
\makecell[cl]{keine Angabe}
\end{tabular}

\end{document}
