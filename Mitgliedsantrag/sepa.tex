
\begin{tcolorbox}[
    colback=white,    % Hintergrundfarbe
    colframe=svoblau,  % Rahmenfarbe
    fonttitle=\bfseries,     % Titel fett
    arc=1.5mm,           % Ecken (optional: arc=2mm für gerundet)
    boxrule=0.5mm            % Dicke des Rahmens
]
\subsection*{Zahlungsempfänger}
\begin{tabular}{@{}l l l l}
\makecell[cl]{Name:} &
\makecell[cl]{\TextField[
    name=txtSVOName,
    bordercolor={0 0 0},
    readonly,
    value={Sportverein 1919 Osterburken e.V.},
    width=7cm, height=0.5cm,
  ]{}
} \\[3ex]
\makecell[cl]{Straße und Hausnr.:} &
\makecell[cl]{\TextField[
    name=txtSVOStrasseHausnr,
    bordercolor={0 0 0},
    readonly,
    value={Postfach 1244},
    width=7cm, height=0.5cm
  ]{}
} & & \\[3ex]
\makecell[cl]{PLZ und Ort:} &
\makecell[cl]{\TextField[
    name=txtSVOPLZOrt,
    bordercolor={0 0 0},
    readonly,
    value={74702 Osterburken},
    width=7cm, height=0.5cm
  ]{}
} &
\makecell[cl]{Land.} &
\makecell[cl]{\TextField[
    name=txtSVOLand,
    bordercolor={0 0 0},
    readonly,
    value={Deutschland},
    width=4cm, height=0.5cm
]{}}
\end{tabular}
\noindent
\\[2ex]
\begin{tabular}{@{}l l}
\makecell[cl]{Gläubiger-Identifikationsnummer:} &
\makecell[cl]{\TextField[
    name=txtSVOIdentnr,
    bordercolor={0 0 0},
    readonly,
    format={var f = this.getField(event.targetName); f.textFont="Courier";},
    value={ D E 5 9 Z Z Z 0 0 0 0 0 6 1 4 9 5 4},
    width=8cm, height=0.5cm
]{}} \\[2ex]
\makecell[cl]{Mandatsreferenz:\\ (vom Zahlungsempfänger auszufüllen)} &
\makecell[cl]{\TextField[
    name=txtSVOMandatref,
    bordercolor={0 0 0},
    readonly,
    width=8cm, height=0.5cm
]{}}
\end{tabular}

\noindent
\\[0.5ex]
\rule{\textwidth}{0.4pt}
\\[1ex]
Ich ermächtige / Wir ermächtigen (A) den Zahlungsempfänger (Name siehe oben), Zahlungen von meinem / unserem Konto
mittels Lastschrift einzuziehen. Zugleich (B) weise ich mein / weisen wir unser Kreditinstitut an, die vom Zahlungsempfänger
(Name siehe oben) auf mein / unser Konto gezogenen Lastschriften einzulösen.
\par\medskip
Hinweis: Ich kann / Wir können innerhalb von acht Wochen, beginnend mit dem Belastungsdatum, die Erstattung des belasteten
Betrages verlangen. Es gelten dabei die mit meinem / unserem Kreditinstitut vereinbarten Bedingungen.

\noindent
\\[0.5ex]
\begin{tabular}{@{}l l @{\hspace{0.5em}}l}
\makecell[cl]{Zahlungsart:} &
\makecell[cl]{\CheckBox[
    name=checkSVOZahlungsart,
    bordercolor={0 0 0},
    readonly,
    checked
]{}} &
\makecell[cl]{Wiederkehrende Zahlung}
\end{tabular}
\\[1ex]
\rule{\textwidth}{0.4pt}
\\[2ex]
\begin{tabular}{@{}l l @{\hspace{0.5em}}l l @{\hspace{0.5em}}l}
\makecell[cl]{Zahlungspflichtiger:} &
\makecell[cl]{\ChoiceMenu[
    radio,
    radiosymbol=\ding{108},
    name=checkSVOZahler,
    keystroke={selectPayor();},
    bordercolor={0 0 0}
]{}{=A}} &
\makecell[cl]{Antragsteller/-in} &
\makecell[cl]{\ChoiceMenu[
    radio,
    radiosymbol=\ding{108},
    name=checkSVOZahler,
    keystroke={selectPayor();},
    bordercolor={0 0 0}
]{}{=N}} &
\makecell[cl]{Andere Person}
\end{tabular}

\noindent
\\[1ex]
\personneutral{Zahlungspflichtige/r und Bankdaten}{Zahler}
\noindent
\\[2ex]
\address{Zahler}

\noindent
\\[0.5ex]
\rule{\textwidth}{0.4pt}
\\[2ex]
\begin{tabular}{@{}p{3cm} l}
\makecell[cl]{IBAN:\\ (max. 34 Stellen)} &
\makecell[cl]{\IBANField{txtIBANZahler}}
\end{tabular}

\noindent
\\[2ex]
\begin{tabular}{@{}p{3cm} l p{7cm} }
\makecell[cl]{BIC:\\ (8 oder 11 Stellen)} &
\makecell[cl]{\TextField[
    name=txtBICZahler,
    bordercolor={0 0 0},
    maxlen=11,
    format={var f = this.getField(event.targetName); f.textFont="Courier";},
    width=4cm, height=0.7cm
]{}} &
\makecell[cl]{\small Hinweis: Die Angabe des BIC kann entfallen,\\
\small wenn der Zahlungsdienstleister des/der Zahlungspflichtigen\\
\small in einem EU-/EWR-Mitgliedsstaat ansässig ist.}
\end{tabular}
\rule{\textwidth}{0.4pt}
\\[2ex]
\signature{Unterschreiber}

\end{tcolorbox}
