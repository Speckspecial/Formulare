\documentclass[a4paper, 10pt]{article}
\pdfminorversion=7

\usepackage[margin=1.5cm]{geometry}
\usepackage{eforms}
%\everyCheckBox{width=10bp, height=10bp}
%\everyTextField{bordercolor={0 0 0}, height=2cm}
%\everyTextField{\BC{1 0 0}\BG{0.9 0.9 0.9}\textFont{Courier}\textSize{10}}

% Globale Einstellungen für ein harmonisches Erscheinungsbild
%\everyRadioButton{radiosymbol=circle}

\begin{document}

\section*{SV 1919 Osterburken e.V.}
\section*{Aufnahmeantrag (Stand 02/2026)}

Hiermit beantrage ich die Mitgliedschaft beim SV~1919~Osterburken~e.V..

\TextField[name=txtEintrittsdatum, format={AFDate_FormatEx("dd.mm.yyyy");}, keystroke={AFDate_KeystrokeEx("dd.mm.yyyy");}, bordercolor={0 0 0}]{Eintrittsdatum:}

\par\medskip

\begin{tabular}{@{}l l@{}}
\ChoiceMenu[name=radioMitgliedschaft, radio, radiosymbol=\ding{55}, onclick={this.getField("Einzelmitglied").display = display.visible;}, bordercolor={0 0 0}]{}{=EinzelErwachsener} & 
Mitgliedschaft für mich \\
\ChoiceMenu[name=radioMitgliedschaft, radio, radiosymbol=\ding{55}, onclick={this.getField("Einzelmitglied").display = display.hidden;}, bordercolor={0 0 0}]{}{=EinzelKind} & 
Mitgliedschaft für mein Kind (Antragsteller/in ist erziehungsberechtigt.)\\
\ChoiceMenu[name=radioMitgliedschaft, radio, radiosymbol=\ding{55}, onclick={this.getField("Einzelmitglied").display = display.hidden;}, bordercolor={0 0 0}]{}{=Familie} & 
Familienmitgliedschaft für mich, Partner/Partnerin und Kinder \\
\end{tabular}

\subsection*{Antragsteller/Antragstellerin}

\begin{tabular}{@{}l l@{}}
Vorname: & \TextField[name=txtAntragstellerVorname, bordercolor={0 0 0}, width=10cm]{} \\
Nachname: & \TextField[name=txtAntragstellerNachname, bordercolor={0 0 0}, width=10cm]{} \\
Geburtsdatum: & \TextField[name=txtAntragstellerGeburtsdatum, format={AFDate_FormatEx("dd.mm.yyyy");}, keystroke={AFDate_KeystrokeEx("dd.mm.yyyy");}, bordercolor={0 0 0}]{} \\
\end{tabular}

\par\medskip

\ChoiceMenu[radio,name=radioGeschlecht, radiosymbol=\ding{52}, bordercolor={0 0 0}]{Geschlecht:}
{männlich=m, weiblich=w, divers=d}

\subsection*{Kontaktdaten des Antragstellers/der Antragstellerin}

\par\medskip

\begin{tabular}{@{}l l@{}}
\TextField[name=txtStrasse, width=10cm, bordercolor={0 0 0}]{Straße:} & \TextField[name=txtHausNr, width=2cm, bordercolor={0 0 0}]{Hausnr.:} \\
\end{tabular}

\par

\begin{tabular}{@{}l l@{}}
\TextField[name=txtPLZ, width=4cm, bordercolor={0 0 0}]{PLZ:} & \TextField[name=txtWohnort, width=8cm, bordercolor={0 0 0}]{Wohnort:} \\
\end{tabular}

\par\medskip

\begin{tabular}{@{}l l@{}}
Telefonnr: & \TextField[name=txtTelefon, width=8cm, bordercolor={0 0 0}]{} \\
E-Mail: & \TextField[name=txtEMail, width=8cm, bordercolor={0 0 0}]{} \\
\end{tabular}

\iffalse
\Annot[
    name=Einzelmitglied,
    width=\linewidth,
    height=180pt,
    bordercolor={1 1 1}
]{
    \parbox{\linewidth}{
        \section*{Einzelmitglied: Auswahl der Abteilung}

        \begin{tabular}{l l}
        \radioButton{txtAntragstellerAbteilung}{12bp}{12bp}{false} Fußball & \radioButton{txtAntragstellerAbteilung}{12bp}{12bp}{false} Turnen \\
        \end{tabular}
    }
}
\fi

\end{document}
