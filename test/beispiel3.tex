\documentclass{article}

\usepackage[unicode]{hyperref}
\usepackage{ocgx2}      % erzeugt OCG-Layer in LaTeX
\usepackage{eforms}     % Buttons
\usepackage{insdljs}    % eingebettetes JavaScript

\begin{insDLJS}[toggleOCG]{toggleOCG}{}
function toggleOCGByName(layerName) {
    var ocgs = this.getOCGs();
    if (!ocgs) return;

    for (var i = 0; i < ocgs.length; i++) {
        ocgs[i].state = !ocgs[i].state;
    }
}
\end{insDLJS}

\begin{document}

\section*{Sektions-Steuerung}

%            if (ocgs[i].name == "Sektion 1") ocgs[i].state = (event.target.value == "s1");
%            if (ocgs[i].name == "Sektion 2") ocgs[i].state = (event.target.value == "s2");

% Radiobuttons definieren (in einer Gruppe, "sectgroup")
% Die "1" am Ende ist der initial sichtbare Layer.

\begin{center}
\PushButton[
  name=btnToggleLayer,
  onclick={toggleOCGByName("Layer 1");}
]{Layer 1 ein-/ausblenden}
\end{center}

\CheckBox[name=cb, keystroke={
    toggleOCGByName("layerone");
}]{Eingabefeld anzeigen}
\ChoiceMenu[radio,name=sectgroup,
keystroke={
    var ocgs = this.getOCGs();
    for (var i=0; i < ocgs.length; i++) {
        ocgs[i].state = !ocgs[i].state;
    }
}]{}{Sektion 1=s1, Sektion 2=s2}

\vspace{1cm}

% Sektion 1 (anfänglich sichtbar, daher 3. Argument {1})
\begin{ocg}{Sektion 1}{s1}{0}
    \section{Inhalt von Sektion 1}
    Dies ist der Text von Sektion 1, der ein- und ausgeblendet wird.
\end{ocg}

% Sektion 2 (anfänglich unsichtbar, daher 3. Argument {0})
\begin{ocg}{Sektion 2}{s2}{0}
    \section{Inhalt von Sektion 2}
    Dies ist der Text von Sektion 2.
\end{ocg}

% only needed to get the Javascript running
%\switchocg{s1}{S1}
%\switchocg{s2}{S2}

\end{document}
