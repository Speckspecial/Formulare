\documentclass[a4paper]{article}

\usepackage[unicode]{hyperref}
\usepackage{ocgx2}      % erzeugt OCG-Layer in LaTeX
\usepackage{eforms}     % Buttons
\usepackage{insdljs}    % eingebettetes JavaScript

% ---------------------------------------------------------
% Dokumentweites JavaScript zum Umschalten eines OCG
% ---------------------------------------------------------
\begin{insDLJS}[toggleOCG]{toggleOCG}{}
function toggleOCGByName(layerName) {
    var ocgs = this.getOCGs();
    if (!ocgs) return;

    for (var i = 0; i < ocgs.length; i++) {
        ocgs[i].state = !ocgs[i].state;
    }
}
\end{insDLJS}

\begin{document}

Beispiel: OCG‑Layer + JavaScript‑Toggle

Der folgende Text gehört zu einem **OCG‑Layer**, der mit *ocgx2* erzeugt wird.

---

Inhalt im Layer *Layer 1*

\begin{ocg}{Layer 1}{layerone}{true}
\textbf{Dies ist Inhalt im OCG Layer 1.}
Er wird per Button ein‑ und ausgeblendet.
\end{ocg}

---

Button zum Umschalten

\begin{center}
\PushButton[
  name=btnToggleLayer,
  onclick={toggleOCGByName("Layer 1");}
]{Layer 1 ein-/ausblenden}
\end{center}

\switchocg{layerone}{}

\end{document}
