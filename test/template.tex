\documentclass{article}
\usepackage[unicode]{hyperref}
\listfiles

\begin{document}

\section*{Steuerung der Sections}

% --- Buttons für Section A ---
\textbf{Section A steuern:}\par
\PushButton[
  onclick={this.getField("secA").display = display.visible;}
]{A anzeigen}
\quad
\PushButton[
  onclick={this.getField("secA").display = display.hidden;}
]{A ausblenden}

\medskip

% --- Buttons für Section B ---
\textbf{Section B steuern:}\par
\PushButton[
  onclick={this.getField("secB").display = display.visible;}
]{B anzeigen}
\quad
\PushButton[
  onclick={this.getField("secB").display = display.hidden;}
]{B ausblenden}

\bigskip
\hrule
\bigskip

% ============================================================
% SECTION A (eigenes PDF-Objekt)
% ============================================================

\Annot[
  name=secA,
  width=\linewidth,
  height=180pt,
  bordercolor={1 1 1}
]{%
  \parbox{\linewidth}{
    \section*{Section A}

    Dies ist der Inhalt von Section A.  
    Diese Section kann unabhängig von Section B ein- oder ausgeblendet werden.

    \medskip
    Eingabefeld A:

    \TextField[
      name=eingabeA,
      width=0.9\linewidth,
      charsize=12pt
    ]{}
  }
}

\bigskip

% ============================================================
% SECTION B (eigenes PDF-Objekt)
% ============================================================

\Annot[
  name=secB,
  width=\linewidth,
  height=180pt,
  bordercolor={1 1 1}
]{%
  \parbox{\linewidth}{
    \section*{Section B}

    Dies ist der Inhalt von Section B.  
    Auch diese Section ist unabhängig steuerbar.

    \medskip
    Eingabefeld B:

    \TextField[
      name=eingabeB,
      width=0.9\linewidth,
      charsize=12pt
    ]{}
  }
}

\end{document}
